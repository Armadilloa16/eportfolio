\documentclass{article}
\usepackage[left=2.3cm, right=2.3cm, top=1.7cm, bottom=1.7cm]{geometry}



\usepackage{tikz}
\usepackage{mhchem}



\usepackage{Sweave}
\begin{document}
\input{practice_test-concordance}

{\Large \textbf{Name:}} \hspace{3cm} \vspace{1cm}


\begin{center}
{\Huge Year 11 Chemistry}
\end{center}

\begin{center}
{\huge Acids and Bases Practice Test}
\end{center}

{\Large

\vspace{0.2cm}
\hspace{1cm}
\textbf{Question 1} 
\vspace{0.2cm}

Carbonic Acid, \ce{H_2CO_3}, is a diprotic acid.
\vspace{0.2cm}

\textbf{(a)} Write two equations to show the stages of its ionisation.

\begin{center}
  \ce{}


\textbf{(b)} Label the conjugate acid-base pairs in the equations above.
\vspace{0.2cm}

\textbf{(c)} State one substance from the equations above that is amphiprotic.

\vspace{0.8cm}
\begin{center}
\begin{tikzpicture}[scale=0.8]
\draw[black,thin] (0,0) -- (20, 0);
\end{tikzpicture}
\end{center}
\vspace{0.2cm}

\textbf{(d)} Explain how \ce{H_2CO_3} is acting as an acid in one of the equations above.

\vspace{0.8cm}
\begin{center}
\begin{tikzpicture}[scale=0.8]
\draw[black,thin] (0,0) -- (20, 0);
\draw[black,thin] (0,1.5) -- (20, 1.5);
\draw[black,thin] (0,3) -- (20, 3);
\end{tikzpicture}
\end{center}
\vspace{0.2cm}



\vspace{0.2cm}
\hspace{1cm}
\textbf{Question 2} 
\vspace{0.2cm}

Potasium oxide, \ce{K_2O}, is a base. Write its hydrolysis (reaction with water) equation and use this equation to explain why it is a base.
\vspace{0.2cm}

\vspace{0.8cm}
\begin{center}
\begin{tikzpicture}[scale=0.8]
\draw[black,thin] (0,0) -- (20, 0);
\draw[black,thin] (0,1.5) -- (20, 1.5);
\draw[black,thin] (0,3) -- (20, 3);
\draw[black,thin] (0,4.5) -- (20, 4.5);
\end{tikzpicture}
\end{center}
\vspace{0.2cm}

\pagebreak
\vspace{0.2cm}
\hspace{1cm}
\textbf{Question 3} 
\vspace{0.2cm}

\textbf{(a)} In the laboratory, an unknown white powder is suspected to be either a metal or metal carbonate of some kind. When a small amount is added to hydrochloric acid (\ce{HCl}), bubbles of gas are produced rapidly. Describe clearly the procedures you could use to identify the gas and hence rule out that it is either a metal or a metal carbonate.

\vspace{0.9cm}
\begin{center}
\begin{tikzpicture}[scale=0.8]
\draw[black,thin] (0,0) -- (20, 0);
\draw[black,thin] (0,1.5) -- (20, 1.5);
\draw[black,thin] (0,3) -- (20, 3);
\draw[black,thin] (0,4.5) -- (20, 4.5);
\draw[black,thin] (0,6) -- (20, 6);
\draw[black,thin] (0,7.5) -- (20, 7.5);
\draw[black,thin] (0,9) -- (20, 9);
\end{tikzpicture}
\end{center}
\vspace{0.2cm}

\textbf{(b)} Write the equation for the reaction between hydrochloric acid (\ce{HCl}) and magnesium metal (\ce{Mg}).

\vspace{0.8cm}
\begin{center}
\begin{tikzpicture}[scale=0.8]
\draw[black,thin] (0,0) -- (20, 0);
\end{tikzpicture}
\end{center}
\vspace{0.2cm}

\textbf{(c)} Write the equation for the reaction between hydrochloric acid (\ce{HCl}) and magnesium carbonate (\ce{MgCO_3}).

\vspace{0.8cm}
\begin{center}
\begin{tikzpicture}[scale=0.8]
\draw[black,thin] (0,0) -- (20, 0);
\end{tikzpicture}
\end{center}
\vspace{0.2cm}


\pagebreak
\vspace{0.2cm}
\hspace{1cm}
\textbf{Question 4} 
\vspace{0.2cm}

Write balanced equations for these reactions:
\vspace{0.2cm}

\textbf{(a)} Calcium Hydroxide (\ce{Ca(OH)_2}) and sulfuric acid (\ce{H_2SO_4}).

\vspace{0.8cm}
\begin{center}
\begin{tikzpicture}[scale=0.8]
\draw[black,thin] (0,0) -- (20, 0);
\end{tikzpicture}
\end{center}
\vspace{0.2cm}

\textbf{(b)} Ammonia (\ce{NH_3}) and phosphoric acid (\ce{H_3PO_4}).

\vspace{0.8cm}
\begin{center}
\begin{tikzpicture}[scale=0.8]
\draw[black,thin] (0,0) -- (20, 0);
\end{tikzpicture}
\end{center}
\vspace{0.2cm}

\textbf{(c)} Alumina or aluminum oxide (\ce{Al_2O_3}) hydrochloric acid (\ce{HCl}).

\vspace{0.8cm}
\begin{center}
\begin{tikzpicture}[scale=0.8]
\draw[black,thin] (0,0) -- (20, 0);
\end{tikzpicture}
\end{center}
\vspace{0.2cm}

\pagebreak
\vspace{0.2cm}
\hspace{1cm}
\textbf{Question 5} 
\vspace{0.2cm}

Describe clearly, with the use of a diagram to aid if helpful, why you can have dilute carbonic acid (\ce{H_2CO_3}) solution and concentrated carbonic acid solution, even though carbonic acid is said to be a "weak" acid.

\vspace{0.9cm}
\begin{tikzpicture}[scale=0.8]
\draw[black,thin] (0,0) -- (20, 0);
\draw[black,thin] (0,1.5) -- (20, 1.5);
\draw[black,thin] (0,3) -- (20, 3);
\draw[black,thin] (0,4.5) -- (20, 4.5);
\draw[black,thin] (0,6) -- (20, 6);
\draw[black,thin] (0,7.5) -- (20, 7.5);
\draw[black,thin] (0,9) -- (20, 9);
\draw[black,thin] (0,10.5) -- (20, 10.5);
\draw[black,thin] (0,12) -- (20, 12);
\end{tikzpicture}
\vspace{0.2cm}

\pagebreak
\vspace{0.2cm}
\hspace{1cm}
\textbf{Question 6} 
\vspace{0.2cm}

Calculate the pH of the following:
\vspace{0.2cm}

\textbf{(a)} A solution in which the concentration of \ce{H_3O^+} is $10^{-4}$M.

\vspace{0.8cm}
\begin{center}
\begin{tikzpicture}[scale=0.8]
\draw[black,thin] (0,0) -- (20, 0);
\end{tikzpicture}
\end{center}
\vspace{0.2cm}

\textbf{(b)} A solution in which the concentration of \ce{H_3O^+} is $3.2 \times 10^{-5}$M.

\vspace{0.8cm}
\begin{center}
\begin{tikzpicture}[scale=0.8]
\draw[black,thin] (0,0) -- (20, 0);
\end{tikzpicture}
\end{center}
\vspace{0.2cm}

\textbf{(c)} A solution of $1.7 \times 10^{-3}$M sulphuric acid (\ce{H_2SO_4}).

\vspace{0.8cm}
\begin{center}
\begin{tikzpicture}[scale=0.8]
\draw[black,thin] (0,0) -- (20, 0);
\end{tikzpicture}
\end{center}
\vspace{0.2cm}

\textbf{(d)} What assumptions have you made in your calculation in part (c) above? Are these assumptions reasonable in this case?

\vspace{0.8cm}
\begin{center}
\begin{tikzpicture}[scale=0.8]
\draw[black,thin] (0,0) -- (20, 0);
\draw[black,thin] (0,1.5) -- (20, 1.5);
\draw[black,thin] (0,3) -- (20, 3);
\draw[black,thin] (0,4.5) -- (20, 4.5);
\end{tikzpicture}
\end{center}
\vspace{0.2cm}

}
\end{document}
