\documentclass[varwidth=144mm, 12pt]{standalone}

% Multi-page Tables
\usepackage{longtable}

% Math
\usepackage{amsfonts}
\usepackage{amsmath}

% CMU sans serif font.
\usepackage[T1]{fontenc}
\renewcommand*\familydefault{\sfdefault}

% Hyperlinks
\usepackage{hyperref}

\begin{document}
\begin{longtable}{lp{.85\textwidth}}
Code & \textbf{Name} and Key Concepts \\ \hline
& \\ \endhead
%\multicolumn{2}{l}{\gls{ac} Senior Mathematical Methods} \\ 
MMu1t1 & \textbf{Functions and graphs}: Lines, Quadratics, Inverse Proportions, Polynomials, Relations, Translations and Dilations \\
MMu1t2 & \textbf{Trigonometric functions}: Unit Circle, Radians, SOH CAH TOA, Sine Rule, Exact Values, Amplitude/ Period/ Phase, Sum of Angles Identities \\
MMu1t3 & \textbf{Counting and probability}: Binomial Coefficients, Set Complement Intersection and Union, Probability, $P(A\cup{}B) = P(A) + P(B) - P(A\cap{}B)$, Conditional Probability, Independance \\
MMu2t1 & \textbf{Exponential functions}: Index Laws, Fractional Indices, Functions, Asymptotes, Graphs \\
MMu2t2 & \textbf{Arithmetic and geometric sequences and series}: Arithmetic and Geometric Sequences as Recurrence Relations, Limiting Behaviour, and Partial Sum Formulae, Growth and Decay \\
MMu2t3 & \textbf{Introduction to differential calculus} Average Rate of Change, First Principles, Leibniz Notation, Instantaneous Rate of Change, Slope of Tangent, Derivitive of Polynomials, Linearity of Differentiation, Optimisation, Anti-Derivitives, Interpret Position-Time Graphs \\
MMu3t1 & \textbf{Further differentiation and applications}: Define $e$ as $a$ s.t. $\lim_{h \to 0} \frac{a^h - 1}{h} = 1$, Derivitives of $e^x$ $\sin(x)$ and $\cos(x)$, Chain Product and Quotient Rules, Second Derivitives \\
MMu3t2 & \textbf{Integrals}: Integrate Polynomial Exponential and Trigonometric Functions, Linearity of Integration,  Determine Displacement given Velocity, Definite Integrals, Fundamental Theorem of Calculus, (signed) Area Under a Curve \\
MMu3t3 & \textbf{Discrete random variables}: Frequencies, General Properties, Expected Value, Variance, Standard Deviation, Bernoulli and Binomial Distribtions \\
MMu4t1 & \textbf{The logarithmic function}: Logs as Inverse of Exponentials, Log-Scales, Log Function Graphs, Natural Log, $\frac{d}{dx}\ln(x) = \frac{1}{x}$, $\int{\frac{1}{x}dx} = \ln(x) + c$ for $x > 0$ \\
MMu4t2 & \textbf{Continuous random variables and the normal distribution}: Probability Density Function, Cumulative Distribution Function, Probabilites Expected Value, Variance and Standard Deviation as Integrals, Linear Transformation of Random Variables, Normal Distribution using Technology \\
MMu4t3 & \textbf{Interval estimates for proportions} Simple Random Sampling, Bias, Sample Proportion, Normal Approximation to the Binomial Proportion, Wald Confidence Interval, Trade-Off Between Width and Level of Confidence \\
& \\
%\multicolumn{2}{l}{\gls{ac} Senior Specialist Mathematics} \\ 
SMu1t1 & \textbf{Combinatorics} Multiplication of Possibilities, Factorial Notation, Permutations with and without Repeated Objects, Union of Three Sets, Pigeon-Hole Principle, Combinations, Pascals Triangle \\
SMu1t2 & \textbf{Vectors in the plane}: Magnetude and Direction, Scalar Multiplication, Addition and Substraction as a Triangle, Vector Notation, $a\textbf{i} + b\textbf{j}$ Notation, Scalar Dot Product, Projection, Parallel and Perpendicular Vectors \\
SMu1t3 & \textbf{Geometry}: Notation for Implication ($\Rightarrow$) and Equivalence ($\Leftrightarrow$), Converse ($B \Rightarrow A$) Negation ($\neg A \Rightarrow \neg B$) and Contrapositive ($\neg B \Rightarrow \neg A$), Proof by Contradiction, $\forall$ and $\exists$ Notation, Counter-Examples, Circle Theorems, Quadrilateral Proofs in $\mathbb{R}^2$ \\
SMu2t1 & \textbf{Trigonometry}: Graph and Solve Trig Functions, Prove Various Trig Indentities, Reciprocal Trig Functions \\
SMu2t2 & \textbf{Matrices}: Notation, Addition and Scalar Multiplication of Matrices, Multiplicative Identity and Inverse, Determinant, Matrices as Transformations \\
SMu2t3 & \textbf{Real and complex numbers}: Rationality and Irrationality, Induction, $i = \sqrt{-1}$, Complex Numbers $a + bi$ and Arithmetic ($+$, $-$, $\times$, $\div$), Complex Conjugates, Complex Plane,  Complex Conjugate Roots of Polynomials \\
SMu3t1 & \textbf{Complex numbers}: Modulus and Argument, Arithmetic ($\times$, $\div$, and $z^n$) in Polar Form, Convert between Polar and Cartesian Form, De Moivre's Theorem, Roots of Complex Numbers, Factorising Polynomials \\
SMu3t2 & \textbf{Functions and sketching graphs}: Composition of Functions, One-to-One, Inverse Functions, Absolute Value Function, Rational Functions \\
SMu3t3 & \textbf{Vectors in three dimensions}: $a\textbf{i} + b\textbf{j} + c\textbf{k}$ Notation, Equation for Spheres, Parameterised Vector Equations, Equations of Lines, the Cross Product, Equation for a Plane, Systems of Linear Equation (Elimination Method) and Geometric Interpretation of Solutions, Kinematics via Differentiation of Vector Equations, Projectile and Circular Motion \\
SMu4t1 & \textbf{Integration and applications of integration} Substitution, $\int{\frac{1}{x}dx} = \ln{|x|} + c$ for $x \neq 0$, Inverse Trig Functions and their Derivitives, Integrate $\frac{\pm1}{\sqrt{a^2-x^2}}$ and $\frac{a}{a^2 + x^2}$, Partial Fractions, Integration by Parts, Volume of Solids of Revolution, Numerical Integration using Technology \\
SMu4t2 & \textbf{Rates of change and differential equations}: Implicit Differentiation, First-Order Seperable Differential Equations, The Logistic Equation, Kinematics (Rates of Change) \\
SMu4t3 & \textbf{Statistical inference}: Central Limit Theorem and the Resulting Confidence Interval for a Mean \\
& \\
%\multicolumn{2}{l}{\gls{sace} Stage 1 Mathematics} \\ 
S1M1 & \textbf{Functions and graphs}: Equations for a Line, Slope, y-intercept, Intersection of Lines, Reciprocal Function, Asymptotes, Functions vs Relations, Domain, Range, Function Notation \\
S1M2 & \textbf{Polynomials}: Quadratic Equations in Vertex and Factorised Forms, Quadratic Formula, Completing the Square, The Leading Coefficient and Degree of a Polynomials, Cubics, Quartics\\
S1M3 & \textbf{Trigonometry}: Pythagoras, SOH CAH TOA, Cosine Rule, Sine Rule, Unit Circle, Sine and Cosine Functions, Radians, Length of Arc, Area of Sector, Amplitude, Period, Phase, $\tan(x) = \frac{\sin(x)}{\cos(x)}$ \\
S1M4 & \textbf{Counting and statistics}: Factorial, Permutations, Multiplication Principle, Combinations, Discrete vs Continuous Random Variables, Mean, Median, Mode, Range, Interquartile Range, Standard Deviation, Normal Distribution, \\
S1M5 & \textbf{Growth and decay}: Index and Logarithm Laws, Exponential Functions and their Graphs \\
S1M6 & \textbf{Introduction to differential calculus}: Average Rate of Change, First Principles, Notation $f'(x) = \frac{df}{dx}$, $\frac{d}{dx}x^n = nx^{n-1}$, Linearity of Differentiation, Slope of Tangent, Increasing vs Decreasing, Local and Global Maxima and Minima, Stationary Points, Sign Diagram \\
S1M7 & \textbf{Arithmetic and geometric sequences and series}: Arithmetic and Geometric Series as Recurrance Relations and Explicit Expressions, Partial Sums, Limiting Behaviour  \\
S1M8 & \textbf{Geometry}: \href{https://prezi.com/view/vpmCHdeFbIV2xlYHhZiK}{Circle Properties}, Proofs (Direct, Contradiction, and Contrapositive) \\
S1M9 & \textbf{Vectors in the plane}: Component (column) vs $ai + bj$ Notation, Length and Direction, Linear Combinations of Vectors, Scalar Dot Product, Projection, Angle Between Two Vectors and Parallel/ Perpendicular, Geometric Proof \\
S1M10 & \textbf{Further Trigonometry}: Sketch Trigonometric Functions with Translations and Dilations, Solve for Angles, Trigonometric Identities, Reciprocal Trigonometric Functions\\
S1M11 & \textbf{Matrices}: Linear Combinations of Matrices, Matrix Multiplication, The Identity, Inverse Matrices, The $2 \times 2$ Inverse, The $2 \times 2$ Determinant, Linear Transformations (including rotations, reflections and composition) \\
S1M12 & \textbf{Real and complex numbers}: Rationals, Irrationals, Interval Notation, Induction, $i = \sqrt{-1}$, Real and Imaginary Components, Complex Conjugates and Arithmetic, Argand Diagram, Modulus, Complex Roots of Polynomals \\
& \\
%\multicolumn{2}{l}{\gls{sace} Stage 2 Mathematical Methods} \\ 
S2MM1 & \textbf{Further differentiation and applications}: S1M6, Chain Product and Quotient Rules, $e = 2.718...$, $\frac{d}{dx}e^x = e^x$,  $\frac{d}{dx}\sin(x) = \cos(x)$, $\frac{d}{dx}\cos(x) = -\sin(x)$, Second Derivatives, Concavity and Points of Inflection \\
S2MM2 & \textbf{Discrete random variables}: Random Variables, Discrete vs Continuous, Probability Functions and Distributions, Properties of Probabilities, Frequency, Expected Value $E[X] = \sum{xp(x)} = \mu_X$, Standard Deviation $\sigma_X = \sqrt{\sum{(x-\mu_X)^2p(x)}}$, Uniform Bernoulli and Binomial Distributions \\
S2MM3 & \textbf{Integral calculus}: Anti-differentiation, If $F'(x) = f(x)$ then $\int{f(x)dx} = F(x) + c$, Reversing Chain Rule for $\int{f(ax + b)dx}$, Linearity of Integration, Finding the Constant of Integration, Area Under the Curve as Upper and Lower Sum Approximations, Definite Integral, Area Between Two Functions and Between a Negative Function and the x-axis, Fundamental Theorem of Calculus,  \\
S2MM4 & \textbf{Logarithmic functions}: Sketching $y = a\ln(b(x-c))$, $\frac{d}{dx}\ln(x) = \frac{1}{x}$, For $x > 0$ $\int{\frac{1}{x}dx} = \ln(x) + c$ \\
S2MM5 & \textbf{Continuous random variables and the normal distribution}: $P(X = x) = 0$, Probability Density Function, $\mu_X = \int_{-\infty}^{\infty}{xf(x)dx}$, $\sigma_X = \int_{-\infty}^{\infty}{(x - \mu_X)^2f(x)dx}$, $f(x) = \frac{1}{\sigma\sqrt{2\pi}}e^{-\frac{1}{2}\left( \frac{x-\mu}{\sigma}\right)^2}$, Standard Normal $Z = \frac{X - \mu}{\sigma}$, Simple Random Sampling, For $X \sim (\mu, \sigma)$ and $X_i \sim iid X$ Sampling Distributions of $S_n = \Sigma_{i = 1}^{n}{X_i}$ $(n\mu,\sigma\sqrt{n})$ and $\bar{X}_n = \frac{S_n}{n}$ $(\mu,\frac{\sigma}{\sqrt{n}})$, If $X$ is Normally Distributed, then so are $S_n$ and $\bar{X}_n$, Central Limit Theorem (CLT) \\
S2MM6 & \textbf{Sampling and confidence intervals}: Confidence Interval for a Mean using CLT $\left(\bar{x} - z^*\frac{s}{\sqrt{n}} \right) \leq \mu \leq \left( \bar{x} + z^*\frac{s}{\sqrt{n}}\right)$, Wald Interval for a Proportion  \\
& \\
%\multicolumn{2}{l}{\gls{sace} Stage 2 Specialist Mathematics} \\ 
S2SM1 & \textbf{Mathematical induction}: Initial Case and Induction Step \\
S2SM2 & \textbf{Complex numbers}: \\
S2SM3 & \textbf{Functions and sketching graphs}: \\
S2SM4 & \textbf{Vectors in three dimensions}: \\
S2SM5 & \textbf{Integration techniques and applications}: \\
S2SM6 & \textbf{Rates of change and differential equations}: \\
& \\
%\multicolumn{2}{l}{UofA MathStart} \\ 
MS1 & \textbf{Numbers \& Functions}: \\
MS2 & \textbf{Linear Functions}: \\
MS3 & \textbf{Quadratic Functions}: \\
MS4 & \textbf{Rational Functions}: \\
MS5 & \textbf{Trigonometry I}: \\
MS6 & \textbf{Trigonometry II}: \\
MS7 & \textbf{Exponential Functions}: \\
MS8 & \textbf{Logarithms}: \\
& \\
%\multicolumn{2}{l}{UofA MathTrack} \\ 
MT1 & \textbf{Polynomials}: \\
MT2 & \textbf{Matrices}: \\
MT3 & \textbf{Vectors and Applications}: \\
MT4 & \textbf{Systems of Linear Equations}: \\
MT6 & \textbf{Differentiation}: \\
MT7 & \textbf{Applications of Differentiation}: \\
MT8 & \textbf{Exponential and Logarithm Functions}: \\
MT9 & \textbf{Integration}: \\
\end{longtable}
\end{document}