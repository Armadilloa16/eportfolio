\documentclass{report}

\title{Classroom Management Handbook}

\begin{document}

\chapter{Introduction and Theory}

This document is written as a reference manual of strategies for classroom behaviour management. The goal is that a teacher could consult this document in order to prompt them on strategies they might like to try implementing in their classroom. In order to facilitate the intended reference manual use of this document, the language used will be deliberately concise, often simply consisting of dot points or key highlights and is not intended to be comprehensive, but rather to give a broad overview, to allow a teacher to choose some strategies to look into further, and relying on that there exist other more in-depth resources that teachers can use to learn more about those strategies once they have an idea of the approach they want to try and take in any given scenario.

Cognitive Load Theory (CLT)

Zone of Proximal Development (ZPD), Constructivism

Operant Conditioning, Behaviourism


\chapter{Preventative Strategies}

General Points:
Preparation (VIDEO: Praise and Preparation, Amy)
Organisation
Structure

Starter Activity

Start well (focus attention, ensure quiet, transitions, struct

Enthusiasm

Maintain 'Flow'

Establish Routines and Procedures, preferably early.

Clearly communicate expectations and instructions

Non-verbals such as being happy to see the students (VIDEO: Talk too much).

Enjoyable and interesting lesson activities

Relevance

Rapport with students

Discussing classroom practices with students

Modelling appropriate behaviours

Mix of different activities

Clear materials (worksheets etc.)

Manage your time.




\chapter{Supportive Strategies}

Praise (VIDEO: Praise and Preparation, Amy)

Use of Names

Show interest

Proximity ("The Whisper Technique")

Eye Contact

Tactical Ignoring

Wait time.
Broken Sentencees --- pause to get attention (VIDEO: Manage that lesson (chemistry year 8)

Challenge Students

Use humour to depotentiate

Acknowledge others good behaviour
(VIDEO: Attention seeking students)

Change the ZPD (Adjust scaffolding?)

\chapter{Corrective Strategies}

Apply Sanctions

Give choice

Arrange to talk privately with student about their misbehaviour

Use of Names

Show interest

Proximity ("The Whisper Technique")

Eye Contact




\end{document}