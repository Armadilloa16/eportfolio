\documentclass{article}

\usepackage{mhchem}

\usepackage{Sweave}
\begin{document}
\input{summary-concordance}

\begin{center}
{\Huge Acids and Bases} \\[6pt]
{\Large Summary Notes}
\end{center}

\section*{Key Concepts}

This is a summary of the concepts covered in this topic.

\begin{itemize}
  \item Acids donate protons, bases accept protons, protons can be written as \ce{H^+}.
  
  \item Acids react with metals to produce hydrogen gas (\ce{H_2}). Acids react with carbonates to form carbon dioxide (\ce{CO_2}).
  
  \item Similarities in the reactions of different acids with metals and bases (including metal oxides, carbonates, and hydroxides) allow products to be predicted from known reactants. You should be able to:
    \begin{itemize}
      \item Predict the products and write
full and ionic equations for reactions
between a given acid and
metal, metal oxide, hydroxide, carbonate, or
hydrogencarbonate.
      \item Undertake stoichiometric calculations for these
reactions.
      \item Highlight the proton transfer between an acid and a base occuring in these reactions by identifying conjugate acid-base pairs and describing the transfer of protons.
    \end{itemize}

  \item Acids can be classified as monoprotic or
polyprotic depending on the number of
protons available for donation. Substances that can either donate or accept a proton are called amphiprotic. Polyprotic acids often have an amphiprotic intermediate ion that forms in water.

  \item Metal oxides are commonly basic, you should be able to:
    \begin{itemize}
      \item Write equations for the reactions with
water of \ce{Na_2O}, \ce{K_2O}, \ce{CaO}, and other similar metal oxides.
    \end{itemize}

  \item The pH scale is a logarithmic scale that
describes the concentration of hydrogen
ions in aqueous solutions, specifically: \ce{pH = -log([H_3O^+])}. You should be able to 
    \begin{itemize}
      \item Use this relationship to calculate the pH of a given solution.
    \end{itemize}
  
  \item Solutions with pH < 7 are acidic, solutions
with pH > 7 are basic, and solutions with
pH = 7 are neutral.

  \item The strength of acids is explained by the
degree of ionisation in aqueous solution. "Strong" acids ionise completely, while "weak" acids do not. 
  
\end{itemize}

% \pagebreak
% \section*{Worked Examples}
% 
% Still need to add some worked examples... 


\pagebreak
\section*{Extension}

These are some additional concepts you could study for this topic, which in combination with the summary above covers all of Topic 5 in the SACE stage 1 chemistry curriculum. 

\begin{itemize}

  \item Something we haven't spoken about but that you could look into is: 
    \begin{itemize}
      \item Draw structural formulae for \ce{CO_2}, \ce{SO_2}, \ce{SO_3}, \ce{H_2SO_3}, \ce{H_2SO_4}, and \ce{H_3PO_4}.
    \end{itemize}
    
  \item Rearrange the relationship \ce{pH = -log([H_3O^+])} to:
    \begin{itemize}
      \item Calculate the concentration of \ce{H_3O^+} in a solution of a  given pH. 
    \end{itemize}
  
  \item Indicators are weak acids or bases where the acidic form is of a
different colour from the basic form.

  \item Neutralisation (reaction of an acid with a base) is an exothermic reaction.

  \item The oxides of non-metals are commonly acidic and generate oxyacids when dissolved in water. \ce{CO_2} dissolves in rainwater to form
carbonic acid, which is a weak acid, giving
rainwater a \ce{pH} of about $5.6$. Oxides of sulfur and nitrogen in the atmosphere can produce rain with a \ce{pH}
below $5.6$.
    \begin{itemize}
      \item Write equations for the reaction of \ce{CO_2}
with water to produce hydrogen ions.
      \item Write equations for the reactions of
oxides of sulfur (\ce{SO_2}, \ce{SO_3}) and nitrogen (\ce{NO}, \ce{NO_2}) with water
that lead to acid rain.
      \item other non-metal oxides also react with water to produce similar oxyacids. Try writing the equation for the reaction of water with \ce{P_4O_10}.
    \end{itemize}
    
\end{itemize}



\end{document}
