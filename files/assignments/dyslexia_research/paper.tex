%!TEX TS-program = xelatex
%!TEX encoding = UTF-8 Unicode
\documentclass[12pt]{article}

\usepackage{xltxtra,fontspec,xunicode}
\defaultfontfeatures{Scale=MatchLowercase}
%\setromanfont[Numbers=Uppercase]{Hoefler Text}
%\setmonofont[Scale=0.90,Ligatures=NoCommon]{Courier}

\setmainfont[BoldFont={OpenDyslexic3-Bold.ttf} ]{OpenDyslexic3-Regular.ttf}
%\setmainfont[
% BoldFont={OpenDyslexic-Bold.otf}, 
% ItalicFont={OpenDyslexic-Italic.otf},
% BoldItalicFont={OpenDyslexic-BoldItalic.otf}
% ]{OpenDyslexic-Regular.otf}

\usepackage{hyperref}
\hypersetup{
    colorlinks=true,       % false: boxed links; true: colored links
    linkcolor=blue,          % color of internal links (change box color with linkbordercolor)
    citecolor=black,        % color of links to bibliography
    filecolor=blue,      % color of file links
    urlcolor=blue           % color of external links
}

\title{Dyslexia}
\author{Lyron Winderbaum}

\begin{document}

\maketitle



Dyslexia is estimated to occur in 15-20\% of the Australian population, around 10\% being diagnosed with dyslexia and the rest being undiagnosed. In the conservative case that it is only 10\%, a random class of 25 students will have \textbf{at least 2 students} with dyslexia, with a probability of over 70\%. In the more realistic case that it is 20\%, as claimed by the \href{https://dyslexiaassociation.org.au/dyslexia-in-australia/}{Australian Dyslexia Association}, this probability rises to over 97\%. 

As such addressing the needs of dyslexic students, both diagnosed and undiagnosed, is important to ensure they have an equal opportunity to learn in our classrooms. Many strategies for addressing the needs of dyslexic students would benefit all students anyway, and as undiagnosed dyslexia can be even more problematic than diagnosed dyslexia because of students propensity to blame their difficulties in completing tasks on "being stupid" --- a potentially crippling beleif to a learner --- it makes sense to focus on strategies that would be realistic to implement for a whole class and not just for individual diagnosed dyslexic students. It is also important to be realistic and recognise the pressures on teachers time and attention, and as such focus on strategies that are easy to implement and take minimal time and effort from teachers. This is not an ethical motivation --- ethically we would spend as much time and effort as was needed to ensure each and every student in our classes are equally included and have equal access to the learning, but more of a reaslistc consideration in that a single human being only has so much time and attention and they can't do everything.

In this report I will cover a number of strategies for making the learning more accessible to students with dyslexia. Each strategy has a different amount of evidence to demonstrate it's effectiveness, and takes a different amount of time and effort to implement. The strategies I will cover will be broadly grouped under the following sections:
\begin{itemize}
	\item Font Size / Typeface
	\item Use of Colour
	\item Portioning Content / Timing
	\item Tailoring Formatting
\end{itemize}

Then maybe I'll talk a little about the research and literature if I have space.

\section{Font Size / Typeface}

\section{Use of Colour}

\section{Portioning Content / Timing}

\section{Tailoring Formatting}







\end{document}


